\documentclass[12pt, a4paper]{article}
\usepackage[UTF8]{ctex}

% 删除CTeX章节定制,改用titlesec包
\usepackage{titlesec}

% 章节样式定义
\titleformat{\section}[hang]
  {\Large\bfseries\color{darkacademic}}
  {\thesection.}{0.5em}{}
\titlespacing*{\section}
  {0pt}{1em}{0.5em}

\titleformat{\subsection}[hang]
  {\large\bfseries\color{academicblue}}
  {\thesubsection}{0.5em}{}
\titlespacing*{\subsection}
  {0pt}{0.8em}{0.3em}

\usepackage{geometry}
\usepackage{fancyhdr}
\usepackage{graphicx}
\usepackage{amsmath}
\usepackage{amsfonts}
\usepackage{amssymb}
\usepackage{array}
\usepackage{setspace}
\usepackage{enumitem}  % 用于自定义目录编号格式
\usepackage{xcolor}
\usepackage{tikz}
\usepackage{tcolorbox}
\tcbuselibrary{most}
\usepackage[colorlinks=true, linkcolor=darkblue, citecolor=darkblue, urlcolor=blue]{hyperref}
\usepackage{bookmark}
\usepackage{listings}
\usepackage{xcolor}    % 代码高亮配色
\lstset{
    backgroundcolor=\color{lightgray},
    basicstyle=\ttfamily\small,
    keywordstyle=\color{academicblue},
    commentstyle=\color{accentgray},
    stringstyle=\color{red},
    showstringspaces=false,
    breaklines=true,
    numbers=left,
    numberstyle=\tiny,
    frame=single,
    language=C++
}

% 改善自动换行和断词
\usepackage{microtype}  % 改善文字排版质量
\usepackage{xurl}       % 允许在任意位置断开URL和代码
\sloppy                 % 允许更宽松的行间距以避免溢出

% 定义学术风格颜色
\definecolor{academicblue}{RGB}{31,78,121}
\definecolor{lightacademic}{RGB}{214,234,248}
\definecolor{darkacademic}{RGB}{25,25,112}
\definecolor{accentgray}{RGB}{105,105,105}
\definecolor{lightgray}{RGB}{245,245,245}

% 页面设置
\geometry{left=3cm, right=3cm, top=2.5cm, bottom=2.5cm}
% 调整页眉高度,避免 fancyhdr 警告
\setlength{\headheight}{22.2pt}
\pagestyle{fancy}
\fancyhf{}
\renewcommand{\headrulewidth}{0.4pt}
\renewcommand{\headrule}{\hbox to\headwidth{
    \color{academicblue}\leaders\hrule height \headrulewidth\hfill
}}
\fancyhead[L]{\color{academicblue}\fontsize{9pt}{11pt}\selectfont 高级语言程序设计课程实验报告}
\fancyhead[R]{\color{academicblue}\fontsize{9pt}{11pt}\selectfont 李天成 \quad 2451367}
\fancyfoot[C]{\color{academicblue}\fontsize{10pt}{12pt}\selectfont -- \thepage\ --}

% 用 enumitem 自定义目录编号与缩进
\newlist{mytoc}{enumerate}{3}
\setlist[mytoc,1]{label=\arabic*., leftmargin=1.5em, itemsep=0.3em}
\setlist[mytoc,2]{label=\arabic{mytoci}.\arabic*., leftmargin=3em, itemsep=0.2em}
\setlist[mytoc,3]{label=\arabic{mytoci}.\arabic{mytocii}.\arabic*., leftmargin=4.5em, itemsep=0.2em}

\begin{document}

% 封面页
\begin{titlepage}
    \thispagestyle{empty}
    
    \centering

    % 学术机构信息
    \vspace*{2cm}
    
    \begin{minipage}{\textwidth}
        \centering
        
        % 标准学术格式标题
        {\fontsize{16pt}{19pt}\selectfont\bfseries\color{darkacademic}
        高级语言程序设计(基础)课程实验报告}
        
        \vspace{1.5cm}
        
        % 主标题
        {\fontsize{20pt}{24pt}\selectfont\bfseries\color{academicblue}
        虚拟发动机性能监控模块}
        
        \vspace{0.5cm}
        
        {\fontsize{14pt}{17pt}\selectfont\color{accentgray}
        Virtual Engine Performance Monitoring Module}
        
        \vspace{3cm}
        
        % 学术信息表格
        \begin{tabular}{ll}
            \multicolumn{2}{c}{\fontsize{14pt}{17pt}\selectfont\bfseries\color{darkacademic} 作者信息} \\[1cm]
            
            \makebox[4cm][l]{\bfseries\color{academicblue}姓名:} & 李天成 \\[0.8cm]
            \makebox[4cm][l]{\bfseries\color{academicblue}学号:} & 2451367 \\[0.8cm]
            \makebox[4cm][l]{\bfseries\color{academicblue}学院:} & 国豪书院 \\[0.8cm]
            \makebox[4cm][l]{\bfseries\color{academicblue}专业:} & 计算机科学与技术(精英班) \\[0.8cm]
        \end{tabular}
        
        \vspace{3cm}
        
        % 底部信息
        \rule{0.6\textwidth}{0.5pt}
        
        \vspace{0.8cm}        {\fontsize{12pt}{14pt}\selectfont\color{accentgray}
        完成日期:2026年1月4日}
    \end{minipage}
    
\end{titlepage}

% 目录页
\newpage
\setcounter{page}{1}
\begin{center}
    {\Large\bfseries\color{darkacademic} 目录}
\end{center}
\vspace{1cm}
\begin{mytoc}
    \item {\bfseries\color{academicblue} 题目简介}
    \begin{mytoc}
        \item 题目描述
        \item 系统功能要求
        \item 告警逻辑定义
    \end{mytoc}
    \item {\bfseries\color{academicblue} 设计思路与整体架构}
    \begin{mytoc}
        \item 模块化架构设计
        \item 数据流向与处理
        \item 物理仿真模型
        \item 告警状态机设计
    \end{mytoc}
    \item {\bfseries\color{academicblue} 实现细节}
    \begin{mytoc}
        \item 物理引擎与波动模拟
        \item 告警管理与迟滞逻辑
        \item UI渲染与交互系统
        \item 故障注入机制
    \end{mytoc}
    \item {\bfseries\color{academicblue} 遇到的问题及解决方法}
    \begin{mytoc}
        \item 仿真数值的随机游走问题
        \item 临界值附近的告警闪烁
        \item UI渲染的性能优化
    \end{mytoc}
    \item {\bfseries\color{academicblue} 心得体会}
    \begin{mytoc}
        \item 面向对象在仿真系统中的应用
        \item 复杂状态管理的经验
        \item 工业级软件的鲁棒性思考
    \end{mytoc}
\end{mytoc}

\newpage
\section{题目简介}

\subsection{题目描述}
发动机指示与机组告警系统(EICAS, Engine Indication and Crew Alerting System)是现代喷气式客机综合电子显示系统的重要组成部分。它主要用于向飞行员显示发动机的主要参数(如转速、温度、燃油流量等)以及飞机的各种系统告警信息。

本项目旨在基于C++语言和EasyX图形库,开发一个双发喷气式飞机的EICAS仿真系统。该系统不仅需要实时模拟发动机在启动、运行、关机等不同阶段的物理参数变化,还需要具备完善的故障检测与告警功能,能够模拟传感器故障、超温、超速、燃油泄漏等多种异常工况,并根据故障等级(警告、警戒、提示)在界面上给出相应的视觉反馈。

\subsection{系统功能要求}
本仿真系统主要包含以下核心功能:
\begin{enumerate}
  \item \textbf{物理仿真}:模拟双发动机(左发、右发)的N1转速、EGT(排气温度)和燃油系统的实时变化,包含启动过程的对数增长曲线和稳态运行时的随机物理波动。
  \item \textbf{图形界面}:复刻真实的EICAS显示界面,包括模拟仪表盘、数字读数、状态指示灯以及CAS(Crew Alerting System)消息显示区域。
  \item \textbf{故障注入}:支持手动注入14种不同类型的故障,包括单/双传感器失效、燃油泵故障、发动机超温、超速等。
  \item \textbf{智能告警}:根据传感器数据实时判断系统状态,生成不同优先级的告警消息(红色Warning、琥珀色Caution、白色Advisory),并实现告警的抑制与清除逻辑。
\end{enumerate}

\subsection{告警逻辑定义}
系统严格遵循航空工业标准的告警分级原则:
\begin{itemize}
  \item \textbf{Level A (Warning)}:红色,表示需要立即采取行动的紧急情况(如发动机起火、严重超速)。
  \item \textbf{Level B (Caution)}:琥珀色,表示需要飞行员关注并采取行动的异常情况(如油量低、温度偏高)。
  \item \textbf{Level C (Advisory)}:白色,表示系统的状态改变或一般性信息。
\end{itemize}

\section{设计思路与整体架构}

本程序采用了经典的MVC(Model-View-Controller)变体架构,将物理仿真、逻辑控制与界面显示分离,确保了系统的高内聚低耦合。

\begin{figure}[htbp]
    \centering
    \includegraphics[width=0.9\textwidth]{屏幕截图 2026-01-05 000436.png}
    \caption{系统运行界面与架构示意}
    \label{fig:system_overview}
\end{figure}

\subsection{模块化架构设计}
系统主要由以下三个核心模块组成:

\begin{itemize}
  \item \textbf{EngineSimulator (Model)}:负责核心物理引擎的计算。它维护着系统的真实状态(SystemData),处理推力控制、燃油消耗以及故障对物理参数的影响。该模块不依赖于任何UI代码,仅负责数据的生成与更新。
  
  \item \textbf{AlertManager (Controller/Logic)}:负责业务逻辑与告警仲裁。它接收来自Simulator的传感器数据,根据预设的阈值规则进行故障检测,并管理告警消息的生命周期(触发、维持、消除)。该模块实现了迟滞(Hysteresis)逻辑以防止告警闪烁。
  
  \item \textbf{EngineUI (View)}:负责数据的可视化呈现。基于EasyX图形库,它将Simulator的数值和AlertManager的告警状态绘制到屏幕上,并处理用户的鼠标点击事件(如启动/停止、故障注入)。
\end{itemize}

\subsection{数据流向与处理}
系统的主循环遵循“输入-计算-渲染”的帧处理流程:
\begin{enumerate}
    \item \textbf{输入处理}:UI层捕获用户的鼠标操作(如调整推力、注入故障),并将指令传递给Simulator。
    \item \textbf{物理更新}:Simulator根据当前状态(启动/运行)和时间步长(dt),计算下一帧的N1、EGT和燃油数据,并叠加随机波动噪声。
    \item \textbf{告警检测}:AlertManager分析最新的传感器数据,更新告警列表(Active Alerts)。
    \item \textbf{界面渲染}:UI层读取最新的系统数据和告警列表,刷新仪表盘指针、数字显示和CAS消息区域。
\end{enumerate}

\section{实现细节}

\subsection{物理引擎与波动模拟}
本系统的核心是一个基于状态机的物理仿真引擎(EngineSimulator)。为了真实还原航空发动机的运行特性,引擎将运行过程划分为三个独立阶段:启动(Starting)、稳态运行(Running)和关机(Stopping)。

\subsubsection{分阶段模拟算法}
\begin{itemize}
    \item \textbf{启动阶段}:模拟涡扇发动机从静止加速到慢车转速的过程。系统采用对数增长函数模拟 N1 转速和 EGT(排气温度)的上升曲线,体现了物理惯性,避免了数值的突变。
    \item \textbf{稳态运行}:在此阶段,发动机参数由推力杆位置(Thrust Level)决定。系统根据当前的推力设定计算出目标 N1 和 EGT 值。
    \item \textbf{关机阶段}:模拟切断燃油后的惯性旋转停车过程,参数按指数衰减规律下降。
\end{itemize}

\subsubsection{基于目标的随机波动模型}
为了模拟真实机械仪表的“呼吸感”和传感器噪声,系统摒弃了简单的随机游走(Random Walk)算法,因为随机游走容易导致数值随时间发散。本系统采用了“基准值 + 噪声”的稳定模型:
\begin{equation}
    V_{current} = V_{target} \times (1 + \text{Noise}(t))
\end{equation}
其中 $\text{Noise}(t)$ 是一个在 $[-3\%, +3\%]$ 区间内变化的随机因子。为了保证指针运动的平滑性,系统在每帧更新时并非直接跳转到含噪值,而是通过线性插值(Lerp)使当前值平滑趋向目标值,从而实现了既有波动又不会剧烈抖动的逼真效果。

\begin{lstlisting}[caption={基于目标的波动与平滑更新逻辑}]
// EngineSimulator.cpp
void EngineSimulator::updateRunningPhase(double dt) {
    // 1. 计算含噪目标值 (Target + Noise)
    double noisyLeftN1 = addFluctuation(leftTargetN1, Constants::FLUCTUATION_RANGE);
    double noisyLeftEGT = addFluctuation(leftTargetEGT, Constants::FLUCTUATION_RANGE);

    // 2. 执行平滑移动 (Lerp)
    // moveTowards 函数实现了 V_curr = V_curr + (V_target - V_curr) * rate * dt
    systemData_.leftEngine.n1Percentage = moveTowards(
        systemData_.leftEngine.n1Percentage, 
        noisyLeftN1, 
        n1Rate * dt
    );
}
\end{lstlisting}

\subsection{告警管理与迟滞逻辑}
告警管理模块(AlertManager)是系统的“大脑”,负责监控所有传感器数据并生成 CAS 消息。该模块最关键的技术创新在于引入了迟滞(Hysteresis)逻辑。

\subsubsection{迟滞比较器的设计}
在仿真系统中,由于引入了 3\% 的随机波动,当参数值在告警阈值附近(如 $950^\circ C$)波动时,会导致告警状态在“触发”和“消除”之间高频切换,造成界面闪烁。
为此,我们实现了迟滞比较器:
\begin{itemize}
    \item \textbf{触发阈值 ($T_{trigger}$)}:例如,当 EGT 超过 $950^\circ C$ 时,触发 Caution 告警。
    \item \textbf{清除阈值 ($T_{clear}$)}:只有当 EGT 回落到 $950 - \Delta$(如 $935^\circ C$)以下时,才消除告警。
\end{itemize}

代码实现中,通过辅助函数 \texttt{isAlertActive} 检查当前状态,动态调整比较阈值:
\begin{lstlisting}[caption={迟滞比较器实现代码}]
// AlertManager.cpp
void AlertManager::checkTempAbnormal(const EngineData& engine, AlertType type) {
    // 定义迟滞量 (Hysteresis)
    const double HYSTERESIS = 15.0; 
    
    // 获取当前是否已处于告警状态
    bool isAlreadyActive = isAlertActive(type);
    
    // 迟滞逻辑判断
    if (!isAlreadyActive) {
        // 触发条件:超过上限
        if (engine.egt > Constants::EGT_MAX_LIMIT) {
            activateAlert(type);
        }
    } else {
        // 解除条件:低于 (上限 - 迟滞量)
        // 只有当温度显著下降后才消除告警,防止临界值闪烁
        if (engine.egt < (Constants::EGT_MAX_LIMIT - HYSTERESIS)) {
            deactivateAlert(type);
        }
    }
}
\end{lstlisting}

\subsection{UI渲染与交互系统}
用户界面(EngineUI)基于 EasyX 图形库开发,采用了组件化设计思想。

\subsubsection{双缓冲防闪烁技术}
为了解决高频刷新(60FPS)带来的画面闪烁和撕裂问题,系统全面启用了双缓冲技术。在每一帧的渲染周期中:
\begin{enumerate}
    \item 调用 \texttt{BeginBatchDraw()} 开启批量绘图模式,所有绘图指令写入内存显存;
    \item 执行清屏、绘制表盘、绘制指针、绘制文字等操作;
    \item 帧末调用 \texttt{FlushBatchDraw()} 将内存显存一次性拷贝至屏幕。
\end{enumerate}
该机制确保了复杂仪表盘画面的稳定显示。

\begin{lstlisting}[caption={EasyX双缓冲绘图流程}]
// EngineUI.cpp
void EngineUI::initialize() {
    initgraph(Constants::WINDOW_WIDTH, Constants::WINDOW_HEIGHT);
    // 开启双缓冲,防止画面闪烁
    BeginBatchDraw();
    // ... 资源加载 ...
}

void EngineUI::render(const SystemData& data, const std::vector<Alert>& alerts) {
    cleardevice(); // 清空缓冲区
    
    // 绘制各个图层
    drawBackground();
    drawGauges(data);
    drawAlerts(alerts);
    
    // 将缓冲区内容一次性输出到屏幕
    FlushBatchDraw();
}
\end{lstlisting}

\subsubsection{矢量仪表绘制}
系统中的 N1 和 EGT 表盘均采用参数化矢量绘制。通过极坐标转换公式,将归一化的物理数值映射为屏幕坐标:
\begin{equation}
    \begin{cases}
    x = x_0 + r \cdot \cos(\theta) \\
    y = y_0 - r \cdot \sin(\theta)
    \end{cases}
\end{equation}
其中 $\theta$ 根据数值在 $150^\circ$ 至 $360^\circ$ 之间线性插值。此外,表盘的扇形区域颜色会根据当前的告警级别(Normal/Caution/Warning)动态改变,提供直观的视觉反馈。

\subsection{故障注入机制}
系统支持 14 种故障的实时注入,其实现原理是基于“数据源头污染”而非简单的界面欺骗。
\begin{itemize}
    \item \textbf{传感器故障}:通过将 \texttt{SystemData} 中的 \texttt{sensorValid} 标志置为 \texttt{false},导致 UI 层无法获取有效数据,从而显示 "---" 或 "FAIL" 标识。
    \item \textbf{物理故障}:例如“燃油流量过高”故障,并非直接修改显示的流量读数,而是在物理引擎计算层强制增加 \texttt{fuelFlow} 变量。这会引起物理模型的连锁反应——燃油流量增加导致燃烧加剧,进而导致 N1 转速上升和 EGT 温度升高,最终触发多重告警。这种深层模拟保证了故障现象的逻辑自洽性。
\end{itemize}

\begin{lstlisting}[caption={故障注入对物理参数的影响}]
// EngineSimulator.cpp
void EngineSimulator::injectFault(FaultType fault) {
    switch (fault) {
        case FaultType::FUEL_LEAK:
            // 燃油泄漏:直接减少物理模型中的燃油量
            // 这会导致后续计算中剩余油量加速下降
            systemData_.fuelQuantity -= 5.0 * dt; 
            break;
            
        case FaultType::EGT_OVERHEAT:
            // 模拟过热:强制提升目标温度
            // 物理引擎会自动追随这个新的错误目标值
            leftTargetEGT = 1080.0; // 目标值远高于红色阈值(1000),确保触发Danger
            break;
            
        case FaultType::SENSOR_FAIL:
            // 传感器失效:标记数据无效
            // UI层检测到此标记后会显示琥珀色叉号
            systemData_.leftEngine.sensorValid = false;
            break;
    }
}
\end{lstlisting}

\subsection{安全保护与强制停车逻辑}
为了模拟真实飞机的安全保护机制,系统实现了分级响应策略:
\begin{itemize}
    \item \textbf{Warning (黄色)}:仅显示告警,不干预发动机运行。
    \item \textbf{Danger (红色)}:触发紧急停车程序。
\end{itemize}

特别地,针对“手动故障注入”场景,为了方便教学演示,系统实现了智能延迟停车逻辑:当用户手动注入严重故障(如超温)时,系统会允许物理参数继续爬升并超过 Danger 阈值,直到参数真正接近故障设定的极端目标值(如 $1080^\circ C$)时才触发强制停车。这确保了用户能清晰观察到红色告警状态和仪表盘的极端读数。

\begin{lstlisting}[caption={智能强制停车逻辑}]
// main.cpp
if (highestLevel == AlertLevel::DANGER) {
    bool shouldStop = false;
    if (g_simulator->isFaultActive()) {
        // 手动故障模式:等待物理参数完全达到目标值才停车
        // 容差控制:N1 ±2%, EGT ±15度
        if (g_simulator->isFaultTargetReached()) {
            shouldStop = true;
        }
    } else {
        // 自然故障模式:一旦触发Danger立即停车
        shouldStop = true;
    }

    if (shouldStop && !g_simulator->isStopping()) {
        g_logger->recordEvent(timestamp, "CRITICAL: EMERGENCY SHUTDOWN");
        g_simulator->stopEngine();
    }
}
\end{lstlisting}

\section{遇到的问题及解决方法}

\subsection{仿真数值的随机游走问题}
\textbf{问题描述}:在早期的物理引擎实现中,为了模拟仪表的抖动效果,采用了在上一帧数值基础上累加随机增量的算法($V_{t+1} = V_t + \Delta$)。这种简单的随机游走(Random Walk)模型导致了严重的数值漂移现象,长时间运行后,发动机参数会逐渐偏离物理目标值,甚至出现负转速等不合逻辑的情况。

\textbf{解决方法}:重构了波动算法,改为“基于目标的噪声模型”。系统首先根据推力杆位置计算出确定的物理目标值(Target),然后在此基础上叠加 $\pm 3\%$ 的高斯白噪声,最后通过线性插值(Lerp)计算当前帧的显示值。这种方法确保了数值始终收敛于物理真实值附近,既保留了视觉上的真实波动感,又保证了系统的数值稳定性。

\subsection{临界值附近的告警闪烁}
\textbf{问题描述}:当引入随机波动后,如果发动机参数恰好处于告警阈值(如 EGT $950^\circ C$)附近,叠加的噪声会导致数值在阈值上下高频跳动。这导致 CAS 告警消息和仪表盘颜色在“正常”和“警告”状态之间快速切换,产生严重的频闪现象,极大地干扰了用户的视觉体验。

\textbf{解决方法}:在告警逻辑中引入了迟滞(Hysteresis)机制,即施密特触发器原理。为每个关键参数设置触发阈值($T_{high}$)和清除阈值($T_{low}$)。例如,EGT 告警在超过 $950^\circ C$ 时触发,但必须回落到 $935^\circ C$ 以下才会消除。这 $15^\circ C$ 的安全缓冲区(Buffer Zone)有效过滤了仿真噪声引起的误触发,确保了告警状态的稳定性。

\subsection{UI布局与交互设计的迭代}
\textbf{问题描述}:在初始设计中,UI 布局存在多处不合理:
\begin{itemize}
    \item 按钮布局混乱,缺乏逻辑分组,导致操作不便。
    \item 状态指示灯(RUN Light)逻辑简单,仅判断 $N1 > 95\%$ 即点亮,导致在临界点附近频繁闪烁。
    \item 故障测试按钮的标签与实际触发逻辑不匹配(如标签显示 $>850$ 但实际触发 $>900$),造成测试困惑。
    \item 故障注入功能最初仅通过少数几个按钮循环切换,无法精确控制特定故障,不利于测试。
\end{itemize}

\textbf{解决方法}:
\begin{itemize}
    \item \textbf{网格化布局与功能拆分}:将故障注入功能从简单的循环切换重构为全功能的控制面板。设计了 $4 \times 4$ 的按钮网格,将 14 种具体的故障类型(如单发传感器失效、双发失效、燃油泄漏、各级超温等)独立拆分为单独的按钮,并按功能模块(传感器、燃油、转速、温度)进行逻辑分组,极大提升了测试效率。
    \item \textbf{指示灯迟滞}:为 RUN 指示灯添加迟滞逻辑(点亮阈值 $95\%$,熄灭阈值 $90\%$),确保状态显示的稳定性。
    \item \textbf{逻辑校准}:全面审查并修正了所有测试按钮的触发逻辑,确保按钮标签(如 "ST > 950")与后台注入的故障参数($980^\circ C$)及告警系统的阈值严格对应。
\end{itemize}

\subsection{指针运动的平滑处理}
\textbf{问题描述}:在引入随机波动后,如果直接将计算出的含噪数值赋给仪表盘指针,指针会出现剧烈的机械跳动,这不符合真实物理仪表的阻尼特性,视觉效果非常生硬。

\textbf{解决方法}:在 UI 渲染层引入了线性插值(Linear Interpolation, Lerp)算法。系统不再直接显示当前的物理计算值,而是维护一个“显示值”变量,每帧以一定的速率(如 $0.1$ 的插值系数)向“物理目标值”逼近。
\begin{equation}
    V_{display} = V_{display} + (V_{target} - V_{display}) \times \alpha
\end{equation}
这种处理模拟了机械指针的惯性和阻尼,使得指针在响应数值变化时既灵敏又平滑,完美复刻了真实航空仪表的动态质感。

\subsection{告警阈值的物理合理性修正}
\textbf{问题描述}:初期设定的 EGT 告警阈值($850^\circ C$)过低。在正常巡航推力(N1 95%)下,物理引擎计算出的 EGT 已接近 $850^\circ C$,叠加波动后会导致正常飞行时误报 Warning。

\textbf{解决方法}:参考真实航空发动机数据,调整了告警逻辑。将 Warning 阈值提升至 $950^\circ C$,Danger 阈值提升至 $1000^\circ C$。这既保留了对异常工况的敏感度,又为正常的大推力运行留出了合理的物理裕度(Margin),消除了误报警现象。

\section{心得体会}

\subsection{跨学科知识的融合与应用}
本次实验不仅仅是一次编程练习,更是一次跨学科的学习之旅。为了实现逼真的 EICAS 仿真,我深入查阅了航空发动机的相关资料,了解了 N1 转速、EGT 排气温度、燃油流量等参数之间的物理关联,以及它们在启动、慢车、巡航等不同阶段的变化规律。这让我深刻体会到,优秀的软件工程师不仅需要掌握编程技术,更需要具备快速学习和理解业务领域知识(Domain Knowledge)的能力。只有真正理解了业务背后的物理模型,才能写出逻辑严密、仿真度高的代码。

\subsection{工程规范与鲁棒性设计}
在开发告警系统的过程中,我切身感受到了工业级软件对“鲁棒性”的严苛要求。简单的阈值判断在理想环境下或许可行,但在充满噪声和波动的真实(或仿真)环境中,必须引入迟滞逻辑、信号滤波等机制来保证系统的稳定性。此外,告警分级(Warning/Caution/Advisory)的设计也让我理解了人机交互中信息分层的重要性——在紧急情况下,必须通过最直观的颜色和方式将最关键的信息传递给操作者。这些工程规范和设计原则,对于我未来从事任何领域的软件开发都是宝贵的经验。

\subsection{面向对象设计的实践感悟}
通过将系统拆分为 Simulator(物理模型)、AlertManager(逻辑控制)和 EngineUI(视图显示)三个独立模块,我再次验证了高内聚、低耦合架构的优势。这种设计使得我在后期频繁调整告警阈值或修改 UI 布局时,完全不需要改动物理引擎的核心代码,极大地降低了维护成本和出错概率。这次实践让我对面向对象设计原则(SOLID)有了更直观和深刻的理解。

\end{document}
