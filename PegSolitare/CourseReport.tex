\documentclass[12pt, a4paper]{article}
\usepackage[UTF8]{ctex}

% 删除CTeX章节定制,改用titlesec包
\usepackage{titlesec}

% 章节样式定义
\titleformat{\section}[hang]
  {\Large\bfseries\color{darkacademic}}
  {\thesection.}{0.5em}{}
\titlespacing*{\section}
  {0pt}{1em}{0.5em}

\titleformat{\subsection}[hang]
  {\large\bfseries\color{academicblue}}
  {\thesubsection}{0.5em}{}
\titlespacing*{\subsection}
  {0pt}{0.8em}{0.3em}

\usepackage{geometry}
\usepackage{fancyhdr}
\usepackage{graphicx}
\usepackage{amsmath}
\usepackage{amsfonts}
\usepackage{amssymb}
\usepackage{array}
\usepackage{setspace}
\usepackage{enumitem}  % 用于自定义目录编号格式
\usepackage{xcolor}
\usepackage{tikz}
\usepackage{tcolorbox}
\tcbuselibrary{most}
\usepackage[colorlinks=true, linkcolor=darkblue, citecolor=darkblue, urlcolor=blue]{hyperref}
\usepackage{bookmark}

% 改善自动换行和断词
\usepackage{microtype}  % 改善文字排版质量
\usepackage{xurl}       % 允许在任意位置断开URL和代码
\sloppy                 % 允许更宽松的行间距以避免溢出

% 定义学术风格颜色
\definecolor{academicblue}{RGB}{31,78,121}
\definecolor{lightacademic}{RGB}{214,234,248}
\definecolor{darkacademic}{RGB}{25,25,112}
\definecolor{accentgray}{RGB}{105,105,105}
\definecolor{lightgray}{RGB}{245,245,245}

% 页面设置
\geometry{left=3cm, right=3cm, top=2.5cm, bottom=2.5cm}
% 调整页眉高度,避免 fancyhdr 警告
\setlength{\headheight}{22.2pt}
\pagestyle{fancy}
\fancyhf{}
\renewcommand{\headrulewidth}{0.4pt}
\renewcommand{\headrule}{\hbox to\headwidth{
    \color{academicblue}\leaders\hrule height \headrulewidth\hfill
}}
\fancyhead[L]{\color{academicblue}\fontsize{9pt}{11pt}\selectfont 高级语言程序设计课程实验报告}
\fancyhead[R]{\color{academicblue}\fontsize{9pt}{11pt}\selectfont 李天成 \quad 2451367}
\fancyfoot[C]{\color{academicblue}\fontsize{10pt}{12pt}\selectfont -- \thepage\ --}

% 用 enumitem 自定义目录编号与缩进
\newlist{mytoc}{enumerate}{3}
\setlist[mytoc,1]{label=\arabic*., leftmargin=1.5em, itemsep=0.3em}
\setlist[mytoc,2]{label=\arabic{mytoci}.\arabic*., leftmargin=3em, itemsep=0.2em}
\setlist[mytoc,3]{label=\arabic{mytoci}.\arabic{mytocii}.\arabic*., leftmargin=4.5em, itemsep=0.2em}

\begin{document}

% 封面页
\begin{titlepage}
    \thispagestyle{empty}
    
    \centering

    % 学术机构信息
    \vspace*{2cm}
    
    \begin{minipage}{\textwidth}
        \centering
        
        % 标准学术格式标题
        {\fontsize{16pt}{19pt}\selectfont\bfseries\color{darkacademic}
        高级语言程序设计(基础)课程实验报告}
        
        \vspace{1.5cm}
        
        % 主标题
        {\fontsize{20pt}{24pt}\selectfont\bfseries\color{academicblue}
        孔明棋算法设计与实现}
        
        \vspace{0.5cm}
        
        {\fontsize{14pt}{17pt}\selectfont\color{accentgray}
        Algorithm Design and Implementation of Peg Solitaire}
        
        \vspace{3cm}
        
        % 学术信息表格
        \begin{tabular}{ll}
            \multicolumn{2}{c}{\fontsize{14pt}{17pt}\selectfont\bfseries\color{darkacademic} 作者信息} \\[1cm]
            
            \makebox[4cm][l]{\bfseries\color{academicblue}姓名:} & 李天成 \\[0.8cm]
            \makebox[4cm][l]{\bfseries\color{academicblue}学号:} & 2451367 \\[0.8cm]
            \makebox[4cm][l]{\bfseries\color{academicblue}学院:} & 国豪书院 \\[0.8cm]
            \makebox[4cm][l]{\bfseries\color{academicblue}专业:} & 计算机科学与技术(精英班) \\[0.8cm]
        \end{tabular}
        
        \vspace{3cm}
        
        % 底部信息
        \rule{0.6\textwidth}{0.5pt}
        
        \vspace{0.8cm}        {\fontsize{12pt}{14pt}\selectfont\color{accentgray}
        完成日期:2025年6月17日}
    \end{minipage}
    
\end{titlepage}

% 目录页
\newpage
\setcounter{page}{1}
\begin{center}
    {\Large\bfseries\color{darkacademic} 目录}
\end{center}
\vspace{1cm}
\begin{mytoc}
    \item {\bfseries\color{academicblue} 设计思路与整体架构}
    \begin{mytoc}
        \item 页面组织关系
        \item 页面渲染逻辑
    \end{mytoc}
    \item {\bfseries\color{academicblue} 实现细节}
    \begin{mytoc}
        \item 将页面存为类
        \item 残局模式的实现
        \item 搜索算法的实现
        \begin{mytoc}
            \item 子小标题一
            \item 子小标题二
        \end{mytoc}
    \end{mytoc}
    \item {\bfseries\color{academicblue} 在实验过程中遇到的问题及解决方法}
    \item {\bfseries\color{academicblue} 心得体会}
\end{mytoc}

\section{设计思路与整体架构}

本程序设计借鉴了React JS的函数式编程思想,将整个应用拆分为多个状态节点(StateNode)和组件类,每个节点或组件负责自身的渲染和事件处理,主循环通过状态切换和最小化重绘保证了界面更新的高效与清晰。

\subsection{页面组织关系}
应用入口定义于`Menu.cpp`中的`main`函数,初始化各状态节点的全局对象并启动主循环。页面以状态模式组织:
\begin{itemize}
  \item StateNode基类:定义纯虚方法`render()`和`handleEvent()`。  \item 各具体状态(MainMenuState、ChooseGameState、HowToPlayState、GameState等)继承自StateNode,分别负责菜单、游戏选择、游戏规则、游戏界面等功能。  \item 状态切换由每个节点的`handleEvent()`返回指向下一个StateNode的指针,辅以`StateSwitch.cpp`中的统一调度逻辑。
\end{itemize}
该组织方式使功能模块高度解耦,新增或修改页面只需新增或改写对应的StateNode子类,主框架不受影响。

\subsection{页面渲染逻辑}
本程序采用面向对象的UI组件设计,所有界面元素如按钮(Button)、标题(Title)、棋盘单元(SingleBlock)等均封装为独立类,负责自身的绘制和状态检测。主循环通过不断获取鼠标位置和点击状态,调用当前状态节点的\texttt{handleEvent()}处理交互事件(如按钮点击、棋子选中、提示生成等),并在事件处理后根据组件内部状态自动更新高亮、选中标识。

当组件的状态或页面内容发生变化时,程序会设置\texttt{needsRender}标志,主循环检测到该标志后先调用\texttt{cleardevice()}清空画面,再调用当前状态节点的\texttt{render()}方法,依次调用各UI组件的\texttt{draw()}或\texttt{drawWithHover()}函数进行重绘,最后通过\texttt{FlushBatchDraw()}一并刷新到屏幕。静态页面(如主菜单、帮助界面、确认弹窗)在鼠标移动时即可触发重绘,以响应悬停效果;游戏主界面仅在鼠标移动幅度超过设定阈值、用户点击或游戏状态变化时才重绘,从而避免无谓的重复绘制,实现性能与体验的平衡。

\subsection{架构优点}
本项目结合状态机模式和函数式编程思想,提升了系统的模块化、可维护性和可测试性。

\textbf{状态机设计的优势:}
状态机将每个界面或逻辑节点封装为独立状态类,render()和handleEvent()职责单一,使页面切换逻辑简洁明了。以GameState为例,用户每次点击棋子即触发选择、移动、提示等状态更新,状态机机制可以灵活响应事件,无需修改主循环。模块化设计便于后期维护,新功能或界面只需添加/修改对应状态类,避免代码侵入。

\textbf{函数式编程风格的优点:}
借鉴React的函数式思想,组件渲染逻辑通过纯函数或无副作用的方法实现,减少了隐藏状态和全局变量的使用。UI组件(如Button、Title、SingleBlock)统一由其自身的render()方法绘制,数据和视图分离,便于组合和复用。同时,纯函数特性使单元测试更加容易,提升了代码可读性和可靠性。

\section{实现细节}
本章将详细介绍程序的核心功能实现,包括UI组件化设计、残局回溯逻辑和搜索算法等,以展示本项目在功能模块划分和逻辑实现方面的思路。

\subsection{组件化UI设计}
在UI层面,程序将所有界面元素抽象为组件(Component)实例,每个组件负责自身的渲染和事件响应,典型代表为\texttt{Button}和\texttt{Title}类。此设计借鉴Web前端的CSS样式体系和React的组件化思想:当需要修改按钮或标题样式时,只需在组件内部统一调整,页面中所有实例将即时生效,大大减少了硬编码量并提升了维护效率。
\texttt{Button}对象内部维护位置、尺寸、文本、颜色和启用状态,调用其\texttt{drawWithHover()}方法结合\texttt{isHovered()}检测鼠标悬停,并动态调整边框颜色和线条粗细,实现高亮效果;\texttt{Title}组件则提供自定义字体和字号选项,并根据窗口宽度自动计算文本居中位置,确保在不同分辨率下标题始终水平居中。
% 之后加一张指出页面中title,button的解释图片,以及加上代码块。
\subsection{页面与弹窗类}
所有页面都封装为独立的类,负责管理自身的UI组件和交互逻辑。\texttt{ConfirmBase}类提供了通用的弹窗框架,通过构造函数的参数灵活配置弹窗的标题、正文、提示文本、背景色、弹窗面板颜色、边框颜色以及确认/取消按钮的样式。
具体弹窗状态如\texttt{ExitState}、\texttt{RestartConfirmState}等继承自\texttt{ConfirmBase},它们在构造函数中只需传入不同的文案和配色,并在\texttt{handleEvent()}方法中实现确认或取消操作的跳转逻辑,即可快速生成完整的对话框界面。该设计将弹窗功能高度抽象化,实现了逻辑复用与样式配置的分离,提高了代码的可维护性和可扩展性。
% 加一张页面的图片
\subsection{渲染刷新逻辑优化}
在主循环中,每次无条件重绘都会造成CPU占用率过高,为此引入以下两项优化策略:

(1) 条件渲染:仅在检测到用户交互(如鼠标点击、键盘输入)或页面内容发生变化时,设置渲染标志并调用\texttt{render()}方法;如若连续迭代期间无任何事件或状态变化,则跳过重绘逻辑。

(2) 帧率限制:在每次循环末尾调用\texttt{Sleep(16)},将刷新频率控制在约60帧每秒,这既能保持界面流畅,又能避免CPU空转浪费。

通过以上优化,程序在保证良好用户体验的同时,显著降低了渲染开销,适用于高性能交互式图形应用。



\subsection{搜索算法的实现}
本项目采用迭代加深A*(IDA*)算法作为搜索核心,结合位运算状态编码和跳跃记录结构,实现孔明棋的智能提示功能。

\subsubsection{IDA* 算法原理与流程}
IDA*算法融合深度优先搜索的空间效率与A*的启发式剪枝优势,主要流程如下:
\begin{enumerate}
  \item 首先计算起始状态 $s_0$ 的启发式估价 $h(s_0)$,通过调用 \texttt{heuristic(start)} 实现,具体为 \texttt{std::bitset<64>(s\_0).count() - 1}。例如英文棋盘初始 32 枚棋子时,$h(s_0)=31$。令 \texttt{bound} = $h(s_0)$,并将 \texttt{nextBound} 初始化为 \texttt{INT\_MAX};
  \item 在当前 \verb|bound| 下调用 \verb|dfs(s, g, bound)|,其中参数 $g$ 表示已执行移动步数,$h$ 为 \verb|heuristic(s)| 返回值。访问每个节点时,计算 $f = g + h$。若 $f > bound$,则执行 \verb|nextBound = std::min(nextBound, f)| 并回溯;否则继续扩展该节点;
  \item 若在遍历中到达目标状态(仅剩一枚棋子),则搜索成功并返回路径;否则遍历结束后,将 \verb|bound| 更新为 \verb|nextBound|,进入下一轮迭代;
  \item 重复上述过程,直至找到解或达到超时/步数限制,保证每轮仅扩展满足 $f \le bound$ 的路径,并逐步逼近最优解深度。
\end{enumerate}

\subsubsection{位运算状态编码与移动记录}
为高效表示棋盘状态,程序使用 64 位无符号整数 (\texttt{uint64\_t}) 对棋盘进行编码:
\begin{itemize}
  \item 每个格子对应位图的一位,1 表示有棋子,0 表示空位;
  \item 编码函数 \texttt{encode(st)} 遍历棋盘格子,依据布尔数组构造位图;
  \item \texttt{MoveRecord} 结构体记录一次跳跃的起点、中点和终点索引:
  \begin{verbatim}
struct MoveRecord {
    int fromIndex;    // 起点索引
    int middleIndex;  // 被跳过棋子索引
    int toIndex;      // 终点索引
};
  \end{verbatim}
  \item 程序启动时,调用 \texttt{initMoves(n, coords)} 预生成所有跳跃组合,存入全局容器减少运行时开销。
\end{itemize}

\subsubsection{启发式函数与剪枝策略}
函数 \texttt{heuristic(State s)} 返回当前棋子数减一:
\begin{verbatim}
static int heuristic(State s) {
    return (int)std::bitset<64>(s).count() - 1;
}
\end{verbatim}
该估价满足可承认性和一致性,可在常数时间内估计最少剩余步数。

剪枝策略包括:
\begin{itemize}
  \item 阈值剪枝:当 $f=g+h>bound$ 时,立即回溯并更新 \texttt{nextBound};
  \item 超时控制:记录搜索开始时间,若递归超出预设毫秒数则提前终止;
  \item 重复状态避免:依靠启发式函数减少无效分支,可在路径栈中检测循环。
\end{itemize}

\end{document}